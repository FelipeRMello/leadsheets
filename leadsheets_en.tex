% arara: pdflatex: { shell: on }
% !arara: biber
% !arara: pdflatex
\documentclass[load-preamble+]{cnltx-doc}
\usepackage{leadsheets}
\setcnltx{
  package = leadsheets ,
  authors = Clemens Niederberger ,
  email   = contact@mychemistry.eu ,
  url     = https://github.com/cgnieder/leadsheets/ ,
  add-frame-options = {
    skipabove = .5\baselineskip ,
    skipbelow = .5\baselineskip
  } ,
%   info = ,
  add-cmds = {
    chord , chordname ,
    definesongtitletemplate ,
    ifsongproperty ,
    setchordnames ,
    setleadsheets ,
    songproperty
  } ,
  add-envs = {
    chorus,
    interlude,
    intro,
    song,
    verse
  } ,
  index-setup = { othercode=\footnotesize, level=\section} ,
  makeindex-setup = {  columns=3, columnsep=1em }
}

\defbibheading{bibliography}{\section{References}}
\addbibresource{\jobname.bib}

\usepackage{filecontents}
\begin{filecontents}{\jobname.bib}
@book{book:realbook,
  author    = {Various} ,
  title     = {The Real Book} ,
  volume    = {I} ,
  subtitle  = {C Edition} ,
  publisher = {Hal Leonard Publishing Corporation} ,
  isbn      = {978-0634060380} ,
  date      = {2000-01-01} ,
  edition   = {6th edition}
}
@book{book:newrealbook,
  author    = {Various} ,
  title     = {The New Real Book} ,
  volume    = {I} ,
  subtitle  = {C Edition} ,
  publisher = {Ama Verlag} ,
  isbn      = {978-0961470142} ,
  date      = {2009-02-12} ,
  edition   = {1st edition}
}
\end{filecontents}

\usepackage{booktabs}

\newpackagename\musicsymbols{musicsymbols}
\newpackagename\chordnames{chordnames}

\setchordnames{format=\libertineLF}

\colorlet{property}{cnltxgreen}
\newenvironment{properties}
  {\colorlet{option}{property}\options}
  {\endoptions}

\newrobustcmd*\property[1]{{\colorlet{option}{property}\option*{#1}}}
\newrobustcmd*\symarg[1]{\textcolor{argument}{\code{#1}}}

\AfterPackage{hyperref}{%
\pdfstringdefDisableCommands{%
  \def\property#1{#1}%
}}

\begin{document}

\part{About the Bundle}

\section{License and Requirements}

\license

\leadsheets{} requires the bundles \bnd{l3kernel}~\cite{bnd:l3kernel} and
\bnd{l3packages}~\cite{bnd:l3packages} ro be available.  It also needs the
package \pkg{translations}~\cite{pkg:translations}.

\section{Background}

Over the years I repeatedly wanted to typeset simple leadsheets of songs, \ie,
song lyrics and the corresponding chords\footnote{I also have had the need to
  typeset leadsheets in the style of the \citetitle{book:realbook} -- but this
  is a task where other software than \LaTeX{} is far easier.}.  This is not
too hard with standard \LaTeX{} commands and environments but it is not very
convenient\ldots{} so looking for existing packages is the logical next step
and I found two very promising packages: \pkg{songs}~\cite{pkg:songs} and
\pkg{songbook}~\cite{pkg:songbook}.  However, both were note \emph{quite} what
I wanted.  Just a bit to inflexible in the wrong places, needing tweaking here
and there, and so one.  On the other hand I had quite some code lying on my
hard drive with various attempts of typesetting leadsheets.  This bundle of
packages is now the attempt to have a clean, documented and customisable
version of my code.

\part{The \musicsymbols{} Package}

The \musicsymbols{} package is a very small package that makes the music
symbol font provided by MusiX\TeX{} available as text font and then uses it to
define a number of symbols.  This redefines the macros \cs{sharp}, \cs{flat}
and \cs{natural}.  All defined symbold are listed in
table~\ref{tab:musicsymbols}.

\musicsymbols{} defines three further commands, namely \cs{musix},
\cs{textmusix} -- a font switch and a text font command -- and
\cs{musicsymbol}.  Those commands are meant for internal use only which is why
they're not explained here.

\begin{table}
  \centering
  \newcommand*\showsymbol[1]{\cs{#1}&\csuse{#1}}
  \caption{Symbols defined by \musicsymbols.}
  \begin{tabular}{llll}
    \toprule
      \bfseries command & \bfseries symbol &
      \bfseries command & \bfseries symbol \\
    \midrule
      \showsymbol{sharp}       & \showsymbol{flat} \\
      \showsymbol{doublesharp} & \showsymbol{doubleflat} \\
      \showsymbol{natural} \\
    \midrule
      \showsymbol{trebleclef}  & \showsymbol{bassclef} \\
      \showsymbol{altoclef}    \\
    \midrule
      \showsymbol{allabreve}   & \showsymbol{meterC} \\
    \midrule
      \showsymbol{wholerest}   & \showsymbol{halfrest} \\
      \showsymbol{quarterrest} & \showsymbol{eighthrest} \\
      \showsymbol{sixteenthrest} \\
    \bottomrule
  \end{tabular}
  \label{tab:musicsymbols}
\end{table}

\part{The \chordnames{} Package}
\section{The \cs*{chordname} Command}

\chordnames{} provides the command \cs{chordname}\marg{chord} for convenient
typesetting of chords:

\begin{example}[side-by-side]
  \chordname{Bb7(#9)} \chordname{Bbb9}
  \chordname{C#7(#9)} \chordname{C##9}
\end{example}

\begin{commands}
  \command{chordname}[\marg{chord}]
    Typesetting chords.  Inside the argument every \code{\#} will be replaced
    by $\sharp$ and every \code{b} is replaced with $\flat$.  Numerals and
    parentheses are typeset as superscripts.  Everything between parentheses
    is always a superscript: \verbcode+\chordname{F#7(#11)}+
    \chordname{F#7(#11)}.
\end{commands}

There are several token lists that are treated specially inside \cs{chordname}:
\begin{itemize}
  \item \code{\^{}} -- This token is replaced by \cs*{textsuperscript}.
  \item \code{ma} -- The symbol for major chords.  Per default this is
    empty. \verbcode+\chordname{Gma}+ \chordname{Gma}.
  \item \code{mi} -- The symbol for minor chords.  Per default this is
    \code{m}.  \verbcode+\chordname{Gmi}+ \chordname{Gmi}.
  \item \code{o} -- The symbol for diminished chords.  Per default this is
    \code{\cs*{textsuperscript}\Marg{o}}.  \verbcode+\chordname{Go}+
    \chordname{Go}.
  \item \code{+} -- The symbol for augmented chords.  Per default this is
    \code{\cs*{textsuperscript}\Marg{+}}.  \verbcode!\chordname{G+}!
    \chordname{G+}.
  \item \code{/o} -- The symbol for half diminished chords.  Per default
    this is \code{\cs*{textsuperscript}\Marg{\cs*{o}\Marg{}}}.
    \verbcode+\chordname{G/o}+ \chordname{G/o}.
  \item \code{add} -- This is superscripted: \verbcode+\chordname{Gadd9}+
    \chordname{Gadd9}.
  \item \code{sus} -- This is superscripted: \verbcode+\chordname{Gsus4}+
    \chordname{Gsus4}.
  \item \code{dim} -- This is superscripted: \verbcode+\chordname{Gdim5}+
    \chordname{Gdim5}.
  \item \code{maj7} -- This is superscripted: \verbcode+\chordname{Gmaj7}+
    \chordname{Gmaj7}.
  \item \code{maj9} -- This is superscripted: \verbcode+\chordname{Gmaj9}+
    \chordname{Gmaj9}.
\end{itemize}

How these token lists are treated depends on optional settings:

\begin{example}
  \setchordnames{
    major-seven = $\Delta$ ,
    major-nine  = $\Delta$\textsuperscript{9}
  }
  \chordname{Gmaj7} \chordname{Gmaj9}
  \chordname{G^6} \chordname{G6}
  \chordname{G7^#5}
\end{example}

\section{Options}

Options are set with the command
\begin{commands}
  \command{setchordnames}[\marg{options}]
    where \meta{options} is a comma separated list of keyval options.
\end{commands}
\begin{options}
  \keyval{format}{code}\Default
    Code inserted before a chord within the same group. Can be used for
    special formatting of the chords, with \cs*{sffamily}, say.
  \keyval{sharp}{code}\Default{\cs{sharp}}
    The sharp symbol.
  \keyval{flat}{code}\Default{\cs{flat}}
    The flat symbol.
  \keyval{double-sharp}{code}\Default{\cs{doublesharp}}
    The double sharp symbol.
  \keyval{double-flat}{code}\Default{\cs{doubleflat}}
    The double flat symbol.
  \keyval{aug}{code}\Default{+}
    The augmented symbol.
  \keyval{half-dim}{code}\Default{\cs*{o}\Marg{}}
    The half-diminished symbol.
  \keyval{full-dim}{code}\Default{o}
    The diminished symbol.
  \keyval{dim}{code}\Default{\cs*{textsuperscript}\Marg{dim}}
    The token list \code{dim}.
  \keyval{add}{code}\Default{\cs*{textsuperscript}\Marg{add}}
    The token list \code{add}.
  \keyval{sus}{code}\Default{\cs*{textsuperscript}\Marg{sus}}
    The token list \code{sus}.
  \keyval{major}{code}\Default
    The token list \code{ma}.
  \keyval{minor}{code}\Default{m}
    The token list \code{mi}.
  \keyval{major-seven}{code}\Default{\cs*{textsuperscript}\Marg{maj7}}
    The token list \code{maj7}.
  \keyval{major-nine}{code}\Default{\cs*{textsuperscript}\Marg{maj9}}
    The token list \code{maj9}.
\end{options}

\part{The \leadsheets{} Package}

The \leadsheets{} package allows for easy creation of leadsheets but it also
can be used to create complete songbooks. The basic idea is that songs are
typeset in the \env{song} environment.  Each song gets a number of properties
(see section~\ref{sec:song-properties}) that determine how the title of the
song is typeset.  For the typesetting of the titles a template mechansim is
used (see section~\ref{sec:title-templates}).  Songs can also be tagged.  The
tags then allow to typeset only songs matching a list of tags that is
specified via an option.

\section{The \env*{song} Environment}

\begin{environments}
  \environment{song}[\oarg{options}\marg{properties}]
    The main environment used to typeset songs.
\end{environments}

\subsection{A First Example}
First let's take a look at an example:

\begin{example}[
  compile,
  graphics={trim={2cm 18cm 2cm 2cm},clip},
  add-sourcecode-options={literate=}]
  \documentclass{article}
  \usepackage{leadsheets}
  \begin{document}

  \begin{song}{title=Layla,composer=Eric Clapton,tags={clapton,unplugged,r&b}}
  \begin{verse}
    What will you do when you get lonely? \\
    Noone waiting by your side. \\
    You've been runnin', hidin' much too long. \\
    You know it's just your foolish pride .
  \end{verse}
  \begin{chorus}
    Layla, got me on my knees. \\
    Layla, beggin' darlin', please! \\
    Layla, darlin' won't you ease my worried mind? 
  \end{chorus}
  \end{song}
  
  \end{document}
\end{example}

Per default the songtitle simply is a \cs{section}\sarg{} without any other
additions.  This is the songtitle template \enquote{minimal}, see
section~\ref{sec:title-templates} for more details on those templates and how
to create your own.

\subsection{Using the \env*{song} Environment}

Inside the \env{song} environment a number of additional environments are used
to specify the different parts of a song.  They all are basically the same
kind of environment, namely an \env*{itemize} environment internally where the
only \cs*{item} has the name of the environment as option.  The \env{verse}
environment is a little bit different since verses can be numbered.  If they
are then each usage of \env{verse} inside \env{song} will step a vers number
and print it (as option to the internal \cs*{item}).

\begin{environments}
  \environment{verse}[\oarg{options}]
    An environment for specifying the verses of a song.
  \environment{chorus}[\oarg{options}]
    An environment for specifying the chorus of a song.\\
    This is the same as \beginenv*\Marg{verse}\Oarg{type=chorus,\meta{options}}.
  \environment{intro}[\oarg{options}]
    An environment for specifying the intro of a song.\\
    This is the same as \beginenv*\Marg{verse}\Oarg{type=intro,\meta{options}}.
  \environment{interlude}[\oarg{options}]
    An environment for specifying an interlude of a song.\\
    This is the same as \beginenv*\Marg{verse}\Oarg{type=interlude,\meta{options}}.
  \environment{bridge}[\oarg{bridge}]
    An environment for specifying a bridge of a song.\\
    This is the same as \beginenv*\Marg{verse}\Oarg{type=bridge,\meta{options}}.
\end{environments}

These environments and their options are described in more detail in
sections~\ref{sec:envverse-environment} and~\ref{sec:other-envverse-like}.

\subsection{Options}

The options to the \env{song} environment are the same as the general options
of \leadsheets.  This means you can set the following options either local to
a song or global for the whole document with this command:

\begin{commands}
  \command{setleadsheets}[\marg{options}]
    Setup command for \leadsheets.
\end{commands}

Although I used the word \enquote{global} above \emph{all options are local to
  the current scope}!

\begin{options}
  \keyval{title-template}{template name}\Default{minimal}
    The songtitle template, see section~\ref{sec:title-templates} for
    details.
  \keyval{chord-cs}{cs}\Default{\cs*{chordname}}
    The command that is used to parse the chords.  See
    section~\ref{sec:placing-chords} for details.  \meta{cs} needs to be a
    command that takes a mandatory argument.
  \keyval{song-format}{\TeX{} code}\Default
    \meta{\TeX{} code} is inserted \emph{before} the song title at the
    beginning of the \env{song} environment.
  \keyval{text-format}{\TeX{} code}\Default
    \meta{\TeX{} code} is inserted \emph{after} the song title at the
    beginning of the \env{song} environment.
  \keybool{numbered}\Default{false}
    Determines wether verses are numbered or not.
  \keyval{print-tags}{list of tags}
    A comma separated list of tags. When specified a song will only be printed
    if it is tagged with at least one of the tags in \meta{list of tags}.
\end{options}

\subsection{Song Properties}\label{sec:song-properties}

Songs can have a number of properties which basically are used in songtitle
templates (see section~\ref{sec:title-templates}).  One specific property,
\property{tags}, plays a different role, though.

\begin{properties}
  \keyval*{title}{title}
    This is the main title of the song.
  \keyval*{subtitle}{subtitle}
    A subtitle.
  \keyval*{short-title}{short song title}
    A short title (may be useful in a template that writes the titles in
    \cs*{section}s for a version to be used in the table of contents).
  \keyval*{sort-title}{song title}
    If not set explicitly this property holds the same value as
    \property{title}.
  \keyval*{sort-short-title}{short song title}
    If not set explicitly this property holds the same value as
    \property{short-title}.
  \keyval*{composer}{composer}
    The composer of the song.  As of now this accepts an arbitrary entry but
    maybe this will not be supported any more when indexing will be
    implemented.  No promises.
  \keyval*{sort-composer}{composer}
    If not set explicitly this property holds the same value as
    \property{composer}.
  \keyval*{lyrics}{writer}
    Whoever wrote the lyrics if different from the composer.  As of now this
    accepts an arbitrary entry but maybe this will not be supported any more
    when indexing will be implemented.  No promises.
  \keyval*{sort-lyrics}{writer}
    If not set explicitly this property holds the same value as
    \property{writer}.
  \keyval*{arr}{arranger}
    Whoever arranged the song.  As of now this accepts an arbitrary entry but
    maybe this will not be supported any more when indexing will be
    implemented.  No promises.
  \keyval*{sort-arr}{arranger}
    If not set explicitly this property holds the same value as
    \property{arr}.
  \keyval*{band}{band}
    The band who plays or played the song.
  \keyval*{sort-band}{band}
    If not set explicitly this property holds the same value as
    \property{band}.
  \keyval*{interpret}{interpret}
    The interpret of the song.  As of now this accepts an arbitrary entry but
    maybe this will not be supported any more when indexing will be
    implemented.  No promises.
  \keyval*{sort-interpret}{interpret}
    If not set explicitly this property holds the same value as
    \property{interpret}.
  \keyval*{genre}{genre}
    The genre of the song.
  \keyval*{key}{key}
    The key of the song.  This property will be used when transposing will be
    implemented, see section~\ref{sec:transposing}.
  \keyval*{tempo}{tempo}
    The tempo of the song.
  \keyval*{tags}{tags}
    A comma separated list of tags. Those tags play a role for the option
    {\colorlet{option}{cnltxyellow}\option{print-tags}}.  When that option is
    used a song is only printed if it has at least one of the tags specified
    in the option.
\end{properties}

There are two more properties, \property{counter} and \property{ID} that
cannot be set but are set automatically for each song.  The \property{counter}
simply holds the number of the current song starting from \code{1} for the
first song.  The \property{ID} currently always ia \code{song\meta{counter}}
where \meta{counter} is the current \property{counter} value.

In principle all properties can get list of entries where items are separated
with \code{\visualizespaces{ and }}.  Of course this doesn't make sense for
each property -- a song does only have one title.  But a song can very well
have more than one composer: think of the Beatles where most songs are written
by Paul McCartney and John Lennon\footnote{This is not quite true: most songs
  were written either by Paul or John but legally usually both are the
  composers.}.

It is possibly to define further such properties.  For details see
section~\ref{sec:available-commands}.

\section{The \env*{verse} Environment}\label{sec:envverse-environment}

\begin{environments}
  \environment{verse}[\oarg{options}]
    An environment for specifying the verses of a song.
  \environment{verse*}[\oarg{options}]
    The same as the \env{verse} environment but will always be unnumbered
    regardless of any option settings.
\end{environments}

\begin{example}[
  compile,
  graphics={trim={2cm 18cm 2cm 2cm},clip},
  add-sourcecode-options={literate=}]
  \documentclass{article}
  \usepackage{leadsheets}
  \begin{document}

  \begin{song}[numbered]{title=Foo}
  \begin{verse}
    Lorem ipsum dolor sit amet, consetetur sadipscing elitr,\\
    sed diam nonumy eirmod tempor invidunt ut labore et dolore\\
    magna aliquyam erat, sed diam voluptua.
  \end{verse}
  \begin{verse*}
    Lorem ipsum dolor sit amet, consetetur sadipscing elitr,\\
    sed diam nonumy eirmod tempor invidunt ut labore et dolore\\
    magna aliquyam erat, sed diam voluptua.
  \end{verse*}
  \begin{verse}
    Lorem ipsum dolor sit amet, consetetur sadipscing elitr,\\
    sed diam nonumy eirmod tempor invidunt ut labore et dolore\\
    magna aliquyam erat, sed diam voluptua.
  \end{verse}
  \end{song}
  
  \end{document}
\end{example}

\subsection{Options}

Currently the \env{verse} environment and all related environments only have
one option:

\begin{options}
  \keyval{type}{type}
    The option determines what type of verse is to be typeset. For instance
    \keyis{type}{chorus} will typeset a chorus.  Which types are available
    will become clear in section~\ref{sec:other-envverse-like}.
\end{options}

\begin{example}
  \begin{verse}[type=chorus]
    Lorem ipsum dolor sit amet, consetetur sadipscing elitr,\\
    sed diam nonumy eirmod tempor invidunt ut labore et dolore\\
    magna aliquyam erat, sed diam voluptua.
  \end{verse}
\end{example}

\subsection{Placing Chords}\label{sec:placing-chords}

Inside the \env{song} environment the character \code{\textasciicircum} is an
active character which is defined to place a chord above words.  More
precisely is is equal to the following command:

\begin{commands}
  \command{chord}[\sarg\symarg{-}\marg{chord}\meta{text}\visiblespace]
    Places \meta{chord} centered above \meta{text}.  The starred version
    gobbles the trailing space while the unstarred version does not.  Like the
    star the dash is optional.  It sets the option \option{smash-next-chord}
    to \code{true}.  \meta{text} may be empty but the trailing space
    \emph{must} be there.  If \meta{text} is empty then the chord is place
    above some horizontal space which can be set with the option
    \option{empty-chord-dim}.
\end{commands}

\begin{example}[side-by-side]
  Text \chord{E7}text \chord*{B7}lon ger text 
\end{example}

Note that per default the width of a chord is not ignored:
\begin{example}[side-by-side]
  text \chord{Gbmi7(b5)}text text
\end{example}
However, there is an option which sets the width of a chord to zero:
\begin{options}
  \keybool{smash-chords}\Default{false}
    If set to true the width of the chords set with \cs{chord} is set to
    zero.
  \keybool{smash-next-chord}\Default{false}
    If set to true the width of the next chord set with \cs{chord} is set to
    zero.
\end{options}
\begin{example}[side-by-side]
  \setleadsheets{smash-next-chord=true}
  text \chord{Gbmi7(b5)}text text \par
  text \chord{Gbmi7(b5)}text text \par
  \setleadsheets{smash-chords=true}
  text \chord{Gbmi7(b5)}text text \par
  text \chord{Gbmi7(b5)}text text
\end{example}

Before we forget:
\begin{options}
  \keyval{empty-chord-dim}{dim}\Default{1em}
    The horizontal space that is inserted if the \meta{text} argument of
    \cs{chord} is empty.
\end{options}

While \cs{chord} is available in the whole document the
\code{\textasciicircum} syntax is -- as mentioned before -- only available
inside of the \env{song} environment.

\begin{example}[
  compile,
  graphics={trim={2cm 16cm 2cm 4cm},clip},
  add-sourcecode-options={literate=}]
  \documentclass{article}
  \usepackage{leadsheets}
  \begin{document}

  \begin{song}{title=Layla,composer=Eric Clapton,tags={clapton,unplugged,r&b}}
  \begin{verse}
    ^{C#mi7} What will you do when you get ^*{G#7}lone ly? \\
    ^{C#mi7} Noone ^*{C}wai ting ^{D}by your ^{E}side. ^{E7} \\
    ^{F#mi} You've been ^*{B}run nin', ^*{E}hid in' much too ^{A}long. \\
    ^{F#mi} You know it's ^{B}just your foolish ^{E}pride .
  \end{verse}
  \begin{chorus}
    ^*{A}Lay ---^-{Dmi7}la, \quad ^{Bb} ^{C}got me on my knees. \\
    Lay^-{Dmi7}la, \quad ^{Bb} ^*{C}beg gin' darlin', ^{Dmi7}please, Layla. \\
    Darlin' won't you ease my worried ^{Dmi7}mind? ^{Bb} ^{C}
  \end{chorus}
  \end{song}
  
  \end{document}
\end{example}

\subsection{Transposing}\label{sec:transposing}

Provided a song has the property \property{key} \emph{and} the key is given as
one of the twelve \enquote{usual} keys, \ie, one of the keys given in
table~\ref{tab:key-signatures}, the chords of a song can be transposed.

\begin{options}
  \keyval{transpose}{number}
    Transposes the chords of a song by \meta{number} of semitones.
    \meta{nmuber} can be a negative number, then the chords are transposed
    down.
  \keychoice{enharmonic}{sharp,flat}
    Suppose you transpose a song in the key of \chordname{E} down a semitone.
    \leadsheets{} will then transpose to the key of \chordname{Eb}.  It always
    chooses the key whose signature has less accidentals.  You can force a
    variant, though, by using this option.  With \keyis{enharmonic}{sharp}
    \leadsheets{} would have chosen \chordname{D#} instead of \chordname{Eb}.
\end{options}

\begin{table}[hbp]
  \centering
  \catcode`\#=12
  \catcode`\!=6
  \def\showsignature!1{\chordname{!1}&\code{!1}}%
  \caption{Allowed keys for the \property{key} property.}
  \begin{tabular}{*{8}{l}}
    \toprule
      \bfseries Key & \bfseries Input &
      \bfseries Key & \bfseries Input &
      \bfseries Key & \bfseries Input &
      \bfseries Key & \bfseries Input \\
    \midrule
      \showsignature{C}  && & \showsignature{Ami} \\
    \cmidrule(lr){1-4} \cmidrule(lr){5-8}
      \showsignature{G}  & \showsignature{F}  & \showsignature{Emi}  & \showsignature{Dmi} \\
      \showsignature{D}  & \showsignature{Bb} & \showsignature{Bmi}  & \showsignature{Gmi} \\
      \showsignature{A}  & \showsignature{Eb} & \showsignature{F#mi} & \showsignature{Cmi} \\
      \showsignature{E}  & \showsignature{Ab} & \showsignature{C#mi} & \showsignature{Fmi} \\
      \showsignature{B}  & \showsignature{Db} & \showsignature{G#mi} & \showsignature{Fbmi} \\
      \showsignature{F#} & \showsignature{Gb} & \showsignature{D#mi} & \showsignature{Ebmi} \\
    \bottomrule
  \end{tabular}
  \label{tab:key-signatures}
\end{table}

\begin{example}[
  compile,
  graphics={trim={2cm 16cm 2cm 4cm},clip},
  add-sourcecode-options={literate=}]
  \documentclass{article}
  \usepackage{leadsheets}
  \begin{document}

  \begin{song}[transpose=2]{
      title=Layla,
      composer=Eric Clapton,
      tags={clapton,unplugged,r&b},
      key = Dmi
    }
  \begin{verse}
    ^{C#mi7} What will you do when you get ^*{G#7}lone ly? \\
    ^{C#mi7} Noone ^*{C}wai ting ^{D}by your ^{E}side. ^{E7} \\
    ^{F#mi} You've been ^*{B}run nin', ^*{E}hid in' much too ^{A}long. \\
    ^{F#mi} You know it's ^{B}just your foolish ^{E}pride .
  \end{verse}
  \begin{chorus}
    ^*{A}Lay ---^-{Dmi7}la, \quad ^{Bb} ^{C}got me on my knees. \\
    Lay^-{Dmi7}la, \quad ^{Bb} ^*{C}beg gin' darlin', ^{Dmi7}please, Layla. \\
    Darlin' won't you ease my worried ^{Dmi7}mind? ^{Bb} ^{C}
  \end{chorus}
  \end{song}
  
  \end{document}
\end{example}

\section{Other \env*{verse}-like Environments}\label{sec:other-envverse-like}
Songs can have lots of different kinds of parts: verses, choruses, bridges,
intros, outros, and so on.  Typographically they're all more or less the same,
at least for the purpose of this package.  This means we'd ideally have
environments for all of these parts with a distinct name in order to get a
clean source.  At the same time these environments should all behave basically
the same.  This is what the environments described in the following sections
are for.

\subsection{Available Environments}\label{sec:avail-envir}

\begin{environments}
  \environment{chorus}[\oarg{options}]
    An environment for specifying the chorus of a song.\\
    This is the same
    as \beginenv*\Marg{verse}\Oarg{type=chorus,\meta{options}}.
  \environment{chorus*}[\oarg{options}]
    An environment for specifying the chorus of a song.\\
    This is the same as \beginenv*\Marg{verse}\Oarg{type=chorus*,\meta{options}}.
    It does not display an introductory title to the left of the environment
    text.
  \environment{intro}[\oarg{options}]
    An environment for specifying the intro of a song.\\
    This is the same as \beginenv*\Marg{verse}\Oarg{type=intro,\meta{options}}.
  \environment{intro*}[\oarg{options}]
    An environment for specifying the intro of a song.\\
    This is the same as \beginenv*\Marg{verse}\Oarg{type=intro*,\meta{options}}.
    It does not display an introductory title to the left of the environment
    text.
  \environment{interlude}[\oarg{options}]
    An environment for specifying an interlude of a song.\\
    This is the same as \beginenv*\Marg{verse}\Oarg{type=interlude,\meta{options}}.
  \environment{bridge}[\oarg{bridge}]
    An environment for specifying a bridge of a song.\\
    This is the same as \beginenv*\Marg{verse}\Oarg{type=bridge,\meta{options}}.
  \environment{info}[\oarg{options}]
    An environment for specifying arbitrary information.\\
    This is the same as \beginenv*\Marg{verse}\Oarg{type=info,\meta{options}}.
    It does not display an introductory title to the left of the environment
    text.
  \environment{solo}[\oarg{options}]
    An environment for specifying a solo to a song.\\
    This is the same as \beginenv*\Marg{verse}\Oarg{type=solo,\meta{options}}.
  \environment{solo*}[\oarg{options}]
    An environment for specifying a solo to a song.\\
    This is the same as \beginenv*\Marg{verse}\Oarg{type=solo*,\meta{options}}.
    It does not display an introductory title to the left of the environment
    text.
\end{environments}

\subsection{Own \env*{verse}-like Environments}
All environments mentioned in the previous section were defined with this
command:
\begin{commands}
  \command{newversetype}[\sarg\marg{environment name}\marg{displayed name}]
    This defines an environment \meta{environment name} with the text
    \meta{displayed name} at the start of the environment left to the text.
    The starred version defines an environment which hides the displayed
    name.  The environment also defines a translation string (see
    section~\ref{sec:internationalization}) \code{leadsheets/\meta{environment
      name}} with \meta{displayed name} both as translation fallback and as
    English translation.
\end{commands}



\section{Title Templates}\label{sec:title-templates}

\subsection{Existing Templates}

\subsection{Own Templates}
\subsubsection{An Example}
\begin{example}[
  compile,
  graphics={trim={2cm 18cm 2cm 2cm},clip},
  add-sourcecode-options={literate=}]
  \documentclass{article}
  \usepackage{leadsheets}
  \definesongtitletemplate{custom}{
    \section{%
      \songproperty{title}%
      \ifsongproperty{composer}
        { (by \songproperty{composer})}
        {}%
    }
  }
  \setleadsheets{title-template = custom}
  \begin{document}

  \begin{song}{title=Layla,composer=Eric Clapton,tags={clapton,unplugged,r&b}}
  \begin{verse}
    What will you do when you get lonely? \\
    Noone waiting by your side. \\
    You've been runnin', hidin' much too long. \\
    You know it's just your foolish pride .
  \end{verse}
  \begin{chorus}
    Layla, got me on my knees. \\
    Layla, beggin' darlin', please! \\
    Layla, darlin' won't you ease my worried mind? 
  \end{chorus}
  \end{song}
  
  \end{document}
\end{example}

\subsubsection{Available Commands}\label{sec:available-commands}
First there are two commands related to defining new properties:
\begin{commands}
  \command{definesongproperty}[\marg{property}]
    Defines a new property \meta{property}.  All existing properties have been
    defined this way.  The command can only be used in the preamble.
  \command{copysongproperty}[\marg{from}\marg{to}]
    Copies the values of property \meta{from} to property \meta{to} if
    property \meta{to} has not been set but property \meta{from} has been.
    All \code{sort-\meta{property}} properties have been treated this way.
    The command can only be used in the preamble.
\end{commands}

Then there are a number of commands related to retrieving and using the values
of properties.  All these commands only make sense inside a template
definition.  Some of the commands are expandable which means they can be used
in an \cs*{edef} like context, \ie, there are also suitable for writing the
property values to the table of contents or other auxiliary files.
\begin{commands}
  \expandable\command{songproperty}[\marg{property}]
    Retrieves property \meta{property}.
  \command{printsongpropertylist}[\oarg{code}\marg{property}\marg{between
    two}\marg{between more}\marg{between last two}]\Default{\cs*{@firstofone}}
    Prints a property list \meta{property} separated with \meta{between two}
    if the list contains only two items and separated with \meta{between more}
    and \meta{between last two} if the list contains more than two items.
    \meta{code} is placed directly in front of each item and items are
    surrounded with braces which means that the last token in \meta{code} may
    be a macro with a mandatory argument.
  \command{usesongpropertylist}[\oarg{code}\marg{property}\marg{between}]%
    \Default{\cs*{@firstofone}}
    Like \cs{printsongpropertylist} but separates items with \meta{between}
    regardless of the length of the list.
  \expandable\command{forsongpropertylist}[\marg{property}\marg{code}]
    Places all items of the property list \meta{property} in the input stream,
    each item preceded with \meta{code}.  Items are surrounded with braces
    which means that the last token in \meta{code} may be a macro with a
    mandatory argument.
  \expandable\command{ifsongproperty}[\marg{property}\marg{true}\marg{false}]
    Checks if property \meta{property} has been set.
  \command{ifsongpropertiesequal}[\marg{property 1}\marg{property
    2}\marg{true}\marg{false}]
    Checks if properties \meta{property 1} and \meta{property 2} have been set
    to the same value.
  \command{expandsongpropertycode}[\marg{code}]
    Exhaustively expands \meta{code}.  Experienced user won't need this.  It
    is essentially
    
    \cs*{begingroup}\cs*{edef}\cs*{x}\Marg{\cs*{endgroup}\meta{code}}\cs*{x}.

    (More precisely it is a wrapper for the expl3 function \cs*{use:x}.)  This
    means that any \code{\#} needs to doubled.
\end{commands}

\section{Internationalization}\label{sec:internationalization}

\part{Appendix}
\appendix

% \section{Future Plans}\label{sec:future-plans}

% This very early draft of \leadsheets{} lacks a number of features I have
% planned to implement:

% \begin{itemize}
%   \item Transposing.  The idea is the following: provided the property
%     \code{key} is given to a song an option \option*{transpose} will recieve a
%     positive or negative integer that determines how many half tones a song
%     will be transposed up or down.  Since in equal temperament notes have
%     enharmonic equivalents a further option \option*{enharmonic} will
%     determine if a song with \keyis{key}{E} and \keyis*{transpose}{2} will be
%     transposed to \chordname{F#} or \chordname{Gb}.
%   \item Indexing.  A hopefully flexible indexing mechanism will allow to index
%     songs by title, composers, and interprets.  This will require for a user
%     to be able to set which indexing commands are used for each of those song
%     properties.
% \end{itemize}

\end{document}
